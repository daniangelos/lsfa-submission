\documentclass{entcs}
\usepackage{entcsmacro}
\usepackage{amsmath, amssymb}
\usepackage{framed}


%MY PACKAGES
\usepackage[british]{babel}% Idiomas
\sloppy
% The following is enclosed to allow easy detection of differences in
% ascii coding.
% Upper-case    A B C D E F G H I J K L M N O P Q R S T U V W X Y Z
% Lower-case    a b c d e f g h i j k l m n o p q r s t u v w x y z
% Digits        0 1 2 3 4 5 6 7 8 9
% Exclamation   !           Double quote "          Hash (number) #
% Dollar        $           Percent      %          Ampersand     &
% Acute accent  '           Left paren   (          Right paren   )
% Asterisk      *           Plus         +          Comma         ,
% Minus         -           Point        .          Solidus       /
% Colon         :           Semicolon    ;          Less than     <
% Equals        =3D           Greater than >          Question mark ?
% At            @           Left bracket [          Backslash     \
% Right bracket ]           Circumflex   ^          Underscore    _
% Grave accent  `           Left brace   {          Vertical bar  |
% Right brace   }           Tilde        ~

% A couple of exemplary definitions:

\def\lastname{Angelos \& Nalon}
\begin{document}
\newcommand{\agent}{\ensuremath{a}}
\newcommand{\Agents}{\ensuremath{{\mathcal{A}}}}
\newcommand{\Prop}{\ensuremath{{\mathcal{P}}}}
\newcommand{\wff}{\ensuremath{WFF}\raisebox{-.8ex}{\system{K}{n}{}}}
\newcommand{\sat}[3]{\ensuremath{\langle #1, #2 \rangle \models #3}}
\newcommand{\Model}{\ensuremath{\mathcal{M}}}
\newcommand{\formula}{\ensuremath{\varphi}}
\renewcommand{\iff}{\Leftrightarrow}
\newcommand{\trule}{\ensuremath{\sigma}}
\newcommand{\calculus}[1]{\ensuremath{\mathcal{C}_{#1}}}
\newcommand{\Literals}{\ensuremath{\mathcal{L}}}
\newcommand{\ml}{\ensuremath{i}}
\newcommand{\cprop}[2]{\ensuremath{\ml: \bigvee^{#1}_{#2 = 1}l_{#2}}}
\newcommand{\cneg}{\ensuremath{\ml: l \then \pos{a}m}}
\newcommand{\cpos}{\ensuremath{\ml: l \then \nec{a}m}}
\renewcommand{\stackrel}[1]{\ensuremath{\overset{\text{#1}}{=}}}
\newcommand{\ckn}{\ensuremath{\cal{C}_{\system{K}{n}{}}}}

%\newtheorem{theorem}{Theorem}[section]
%\newtheorem{lemma}[theorem]{Lemma}
%\newtheorem{proposition}[theorem]{Proposition}
%\newtheorem{corollary}[theorem]{Corollary}
%\newtheorem{definition}[theorem]{Definition}

\newenvironment{proof2}[1][Proof]{\begin{trivlist}
\item[\hskip \labelsep {\bfseries #1}]}{\end{trivlist}}
%\newenvironment{example}[1][Example]{\begin{trivlist}
%\item[\hskip \labelsep {\bfseries #1}]}{\end{trivlist}}
%\newenvironment{remark}[1][Remark]{\begin{trivlist}
%\item[\hskip \labelsep {\bfseries #1}]}{\end{trivlist}}


\ifx\fmtname\@psfmtname \else \def\cmsy@{2}\fi % make sure we get cmsy
\def\sometime{\mathord{\hbox{\normalsize$\mathchar"0\cmsy@7D$}}}


\newcommand{\always}{\raisebox{-.2ex}{
			   \mbox{\unitlength=0.9ex
			   \begin{picture}(2.3,2.3)
			   \linethickness{0.06ex}
			   \put(0,0){\line(1,0){2.3}}
			   \put(0,2.3){\line(1,0){2.3}}
			   \put(0,0){\line(0,1){2.3}}
			   \put(2.3,0){\line(0,1){2.3}}
			   \end{picture}}}
		      \,}

\newcommand{\alwaysi}[1]{\raisebox{-.2ex}{
			   \mbox{\unitlength=0.9ex
			   \begin{picture}(2,2)
			   \linethickness{0.06ex}
			   \put(0,0){\line(1,0){2}}
			   \put(0,2){\line(1,0){2}}
			   \put(0,0){\line(0,1){2}}
			   \put(2,0){\line(0,1){2}}
                           \put(0.5,0.5){\scriptsize$#1$}
			   \end{picture}}}}

\newcommand{\nec}[1]{\!\alwaysi{#1}\,}
\newcommand{\pos}[1]{{	   \mbox{\unitlength=0.8ex
			   \begin{picture}(2,2)
			   \linethickness{0.06ex}
			   \put(0,0){$\sometime$}
                           \put(0.6,0.4){\tiny$#1$}
			   \end{picture}}}\,}

%\newcommand{\nec}[1]{{	   \mbox{\unitlength=1.5ex
			   %\begin{picture}(2,2)
			   %\linethickness{0.06ex}
			   %\put(0,0){$\always$}
                           %\put(0.8,0.4){\footnotesize$ #1$}
			   %\end{picture}}}\,}

%\newcommand{\pos}[1]{{	   \mbox{\unitlength=1.5ex
			   %\begin{picture}(1.5,1.5)
			   %\linethickness{0.06ex}
			   %\put(0,0){$\sometime$}
                           %\put(0.6,0.4){\footnotesize$#1$}
			   %\end{picture}}}\,}

%\newcommand{\pos}[1]{\sometime _{#1}}
\newcommand{\then}{\Rightarrow}
\newcommand{\onlyif}{\Leftarrow}
\newcommand{\ifonlyif}{\Leftrightarrow}
\newcommand{\tvalue}[1]{\mbox{\it #1\/}}
\newcommand{\constant}[1]{\mbox{\rm\bf #1}}
\newcommand{\system}[3]{\raisebox{.2ex}[1.2ex]{\raisebox{-.2ex}{{\sf #1}}{$_{#2}^{#3}$}}}
%\newcommand{\system}[3]{{\sf #1}$_{#2}^{#3}$}
\newcommand{\set}[1]{\mbox{$\mathcal{#1}$}}
\newcommand{\WFF}[2]{{\sf WFF}{\mbox{$_{\mbox{\small\sf #1}_{#2}}$}}}
\newcommand{\know}[1]{\mbox{K$_{#1}\,$}}
\newcommand{\knownot}[1]{\mbox{$\nec{#1}\neg\,$}}
\newcommand{\believe}[1]{\mbox{B$_{#1}\,$}}
\newcommand{\relation}[2]{\mbox{$\mathcal #1$$_{#2}$}}
%\newcommand{\universal}{\always^{*}}
\newcommand{\universal}{\always^*}
\newcommand{\snf}[1]{{\sf SNF}\mbox{$_{\mbox{\scriptsize #1}}$}}
\newcommand{\binomial}[2]{\left(\! \begin{array}{c}
                                 #1\\
                                 #2
                                 \end{array}
                          \!\right )}
\newcommand{\ap}[1]{\alpha(#1)}

\newcommand{\ib}{\bf}

\newcommand{\NKN}{\hbox{\it NKN}}
\newcommand{\NEW}{\hbox{\it NEW}}


\newcommand{\Nat}{\mbox{$\mathbb N$}}

%\newcommand{\comment}[1]{}
     
\newcommand{\cfalse}{\constant{false}}
\newcommand{\ctrue}{\constant{true}}
\newcommand{\st}{\ensuremath{w}}
\newcommand{\St}{\ensuremath{W}}
\newcommand{\depth}{\sf profundidade}
\newcommand{\model}[1]{{\cal #1}}


\begin{frontmatter}
    \title{A Clausal Tableaux for Modal Logics} 
    \author{Daniella Angelos\thanksref{myemail}}
    \address{Department of Computer Science\\ University of Bras\'ilia\\
    CEP: 70.910-090, Bras\'ilia, DF, Brazil} 
    \author{Cl\'audia Nalon\thanksref{coemail}}
    \address{Department of Computer Science\\ University of Bras\'ilia\\
    CEP: 70.910-090, Bras\'ilia, DF, Brazil} 
%    \thanks[ALL]{Thanks to everyone who should be thanked} 
    \thanks[myemail]{Email: \href{mailto:angelos@aluno.unb.br} {\texttt{\normalshape angelos@aluno.unb.br}}} 
    \thanks[coemail]{Email: \href{mailto:nalon@unb.br} {\texttt{\normalshape nalon@unb.br}}}
\begin{abstract} 
\end{abstract}
\begin{keyword}
\end{keyword}
\end{frontmatter}
\section{Introduction}
\label{intro}

\section{Language}
\label{sec:language}

The language of \system{K}{n}{} is equivalent to its set of \emph{well-formed
formulae}, denoted \wff, which is constructed from a denumerable set of
\emph{propositional symbols} $\Prop = \{p, q, r, \ldots\}$, the negation
symbol $\neg$, the disjunction symbol $\lor$ and the modal connectives
$\nec{a}$, that express the notion of necessity, for each index $a$
in a finite, fixed set $\mathcal{A} = \{1, \ldots, n\}, n \in \mathbb{N}$.

%The propositional symbols combined with the logic operators are arranged to form sentences (parentheses can be used to avoid ambiguity also). Therefore, the set of \wff~is recursively defined as showed in Definition~\ref{def:wff}.

\begin{definition}
\label{def:wff}
    The set of well-formed formulae, \wff, is the least set such that:
    \begin{enumerate}
        \item $p \in \wff$, for all $p \in \Prop$
            \vspace{.2ex}
        \item if $\varphi, \psi \in \wff$, then so are $\neg \varphi, (\varphi
            \lor \psi)$ and $\nec{a} \varphi$, for each $a \in \Agents$
    \end{enumerate}
\end{definition}

When $n = 1$, we often omit the index in the modal operators, i.e., we just write $\nec{} \varphi$ and $\pos{}\varphi$, for a formula $\varphi$. Other logic operators may be introduced as abbreviations, as usual:
$\varphi \wedge \psi \stackrel{def} \neg(\neg \varphi \lor \neg \psi)$
(conjuction),
$\varphi \then \psi \stackrel{def} \neg \varphi \lor \psi$ (implication),
$\varphi \iff \psi \stackrel{def} (\varphi \then \psi) \land (\psi \then
\varphi)$ (equivalence),
$\pos{a} \varphi \stackrel{def} \neg \nec{a} \neg \varphi$ (possibility),
$\textbf{false} \stackrel{def} \varphi \wedge \neg \varphi$ (\emph{falsum}),
$ \textbf{true} \stackrel{def} \neg \textbf{false}$ (\emph{verum}). Parentheses may be ommitted if the reading is not ambiguous.

A \emph{literal} is a propositional symbol $p \in \Prop$ or its negation $\neg p$. We denote by \Literals~the set of all literals. A \emph{modal literal} is a formula of the form $\nec{a} l$ o $\pos{a} \neg l$, with $l \in \Literals$ and $a \in \Agents$.

%In the following, we often refer to formulae of the form $\nec{a} \varphi$ as \emph{box} \varphi~and to formulae of the form $\pos{a} \varphi$ as \emph{diamond} \varphi.

%Logics that involve $n$ agents in the modal logic, with $n \in \Nat$, are know
%as Multimodal Logics. 

The maximal number of nesting modal operators in a formula is defined as its \emph{modal depth} and denoted $mdepth$. The maximal number of modal operators in which scope the formula occurs is defined as the \emph{modal level} of that formula, and it is denoted $ml$. For instance, in $\nec{a}\pos{a} p$, $mdepth(p) = 0$ and $ml(p) = 2$.

The semantics of \system{K}{n}{} is presented in terms of Kripke structures.

\begin{definition}
    A Kripke model for \Prop~and \Agents~is given by the tuple $\Model = (\St, \st_0, R_1, \ldots, R_n, \pi)$,
    where $\St$ is a non-empty set of possible worlds with a distinguinshed world
    $\st_0$, the root of \Model; each $R_a$, $a \in \Agents$, is a binary relation
    on $\St$, and $\pi: \St \times \Prop \longrightarrow \{false, true\}$ is the
    valuation function that associates to each world $\st \in \St$ a
    truth-assignment to propositional symbols.
\end{definition}

Satisfiability and validity of a formula in a given world is defined as follows.

\begin{definition}
\label{relsat}
    Let $\Model = (\St, \st_0, R_1, \ldots, R_n, \pi)$ be a Kripke model, $\st \in \St$ a world, and $\varphi, \psi \in \wff$. The \emph{satisfiability relation}, denoted by 
    \sat{\Model}{\st}{\varphi}, between a world \st~and a formula $\varphi$,
    is inductively defined by:
    \begin{enumerate}
        \item \sat{\Model}{\st}{p} if, and only if, $\pi(\st, p) = \textbf{true}$, for all $p \in \Prop$;
        \item \sat{\Model}{\st}{\neg\varphi} if, and only if, $
            \langle \Model, \st \rangle \not \models \varphi$;
        \item \sat{\Model}{\st}{\varphi\lor\psi} if, and only if,
            \sat{\Model}{\st}{\varphi} or \sat{\Model}{\st}{\psi}
        \item \sat{\Model}{\st}{\nec{a} \varphi} if, and only if, for all $t\in
            \St$, $(\st, t) \in R_a$ implies  \sat{\Model}{t}{\varphi}
    \end{enumerate}
\end{definition}

Satisfiability is defined in terms of the root of a model. A formula $\varphi \in \wff$ is said to be \emph{satisfiable} if there exists a Kripke model $\Model = (\St, \st_0, R_1, \ldots, R_n, \pi)$ such that \sat{\Model}{\st_0}{\varphi}. A formula is said to be \emph{valid} if it is satisfiable in all models.

The local satisfiability problem in \system{K}{n}{} corresponds to determining the existence of a model at which a formula is satisfied. The local satisfiability problem for \system{K}{n}{} is PSPACE-complete \cite{Spaan:coml}.

\section{A Clausal Tableaux for \protect{\system{K}{n}{}}}
\label{sec:tableauxclausal}

Formally, a proof is a finite object constructed according to fixed set of syntactic rules that refer only to the structure of formulae, not to their intended
meaning. The set of syntactic rules that are used to provide are said to specify a \emph{calculus}. A calculus is \emph{sound} for a particular logic if any
formula that has a proof is a valid formula of this logic, and is
\emph{complete} for a logic if every valid formula has a proof~\cite{fitting}.
From these definitions, complete and sound calculi allow us to produce proofs that formulae are valid in a specific logic. As a formula is valid if, and only if, its negation is unsatisfiable, calculi which are constructed for satisfiability checking can also be used to determine whether a formula is valid or not.

Tableaux-based methods are frequently used in modal
logics because their structure is more obviously related to the notion of
possible worlds \cite{fitting}. In general, proofs are graphically represented by trees where each branch can be thought as a set of formulae. Here we present a tableaux-based calculus for deciding the satisfiability of formulae in \system{K}{n}{}. Unlike many tableaux-based systems, our system is clausal and clauses are labelled by the modal level in which a formula occur. The normal form we use is called \emph{Separated Normal Form with Modal Levels} (\snf{K}) which separates the contexts considering the different modal ineceses and different modal levels appropriately. The transformation rules and the correction proof of the translation method can be find in~\cite{nalon2015modal}.

A formula in this clausal form is represented by a set of clauses, which are
true in their respectives modal levels. A formula in \snf{K} is of the form:
$\bigwedge_i ml : C_i$,
where each $C_i$ is a clause and $ml$ is the modal level in which the clause
occurs. A clause in \snf{K} is in one of the following syntactic forms:
\begin{itemize}
    \item literal clause:
        $
        ml: \bigvee^r_{b=1} l_b
        $
    \item negative $a$-clause:
        $
        ml: l \then \pos{a}m
        $
    \item positive $a$-clause:
        $
        ml: l \then \nec{a}m
        $
\end{itemize}
where $l_b, l, m \in \Literals, a \in \Agents$ and $r,b,ml \in \Nat$. As conjunctions are associative, commutative, and idempotent, we often refer to a formula into \snf{K} as a set of clauses.


The proposed calculus, denoted by \ckn, comprehends a set of inference rules to
deal with both propositional and modal reasoning. Before presenting the inference rules, we need to define some more notation. Let $\set{C}$ be a set of clauses and $i \in \Nat$. We denote by $\gamma^i$ the set of literal clauses in $\set{C}$ occurring at the modal level $i$, that is, the set of clauses of the form $i: \bigvee^r_{b=1} l_b$; by $\lambda^i$ the set of negative $\agent$-clauses in $\set{C}$, that is, the clauses of the form $i : l \then \pos{\agent}m$, for all $\agent \in \Agents$; and by $\theta^i$ the set of positive $\agent$-clauses in $\set{C}$, that is, clauses of the form $i : l \then \nec{a}m$, for all $\agent \in \Agents$. We use capital Greek letters to denote sets of sets of clauses. Initially, $\Gamma^i = \{\gamma_i\}$, $\Lambda^i = \{\lambda^i\}$, and $\Theta^i = \{\theta^i\}$, for all modal levels $i$. We denote by $\Delta^i$ a set of literals and modal literals occurring at the modal level $i$. We set $\Delta^i =\emptyset$, initially. Finally, we denote by $\Pi^i$ all sets of literals occurring at a the modal level $i$.

We now present the inference rules that are aplied to theses sets of literals. The inference rules try to build sets of literals for each modal level, starting with $\Pi^i = \{\emptyset\}$ (because $\Delta^i = \emptyset$ initially). 

The first inference rule, (PROP), is applied to literal clauses. It takes as premisse the sets of literals already built and expands these sets with literals occurring in clauses in $\Gamma_i$, as shown in the conclusion of the rule. The
intuition for this rule is that if $\Delta^i_j$ and $\gamma^\ml \in \Gamma^\ml$
are both satisfiable, then $\Delta^i_j \cup \{l_r\}$ has to be satisfiable for
some $1 \leq r \leq t$, therefore, at least one of the conclusions is
satisfiable.

The rules (NEG) and (POS) are applied to modal clauses. For (NEG), the premise is applied to every set of literals $\Delta^i_j$ already built and to a negative $\agent$-clause. As negative $a$-clauses can be seen as disjunctions, the conclusion is a branching, where the negation of the left-hand side is added to one of the sets and both the left-hand side and the right-hand side is added to the other set. The inference rule (POS) is similar.

The last calculus' rule, (EXP), has as numerator all sets $\Delta^i_k \in
\Pi^i$, at the \ml-th modal level, such that $\{\pos{a}m_0, \nec{a}m_1, \ldots,
\nec{a}m_r\} \subseteq \Delta^i_k$ for some $m_0, m_1, \ldots, m_r \in
\Literals, r \geq 0$. That means that this set contains at least one diamond and
an arbitrary number of boxes.  This rule expresses that if $\Delta^i_k$ is
satisfiable then exists at least one set $\Delta^{i+1}_j \in \Pi^{i+1}$ that
satisfies $\{m_0, m_1, \ldots, m_r\}$, that is, each literal in a diamond must
be satisfied at the next modal level along with all the literals in the boxes in
the same set.

\begin{figure*}
\begin{framed}
\begin{center}
    \small{\
    \begin{tabular}{c}
        
        \ensuremath{\begin{array}{cc}
            \begin{array}{c}
                \Delta^i_j \in \Pi^i \qquad \Gamma^i \cup \{i : \bigvee^t_{k=1} l_k\} \\ \cline{1-1}
                \Delta^i_j \cup \{l_1\}\ |\ \ldots\ |\ \Delta^i_j \cup \{l_t\}
            \end{array}
            &
            \mbox{(PROP)}
        \end{array}
    %\caption{Regra de inferência do tableaux para cláusulas
    %proposicionais}
            }
\vspace{1.5em}
\\

\begin{tabular}{cc}
    \ensuremath{\begin{array}{cc}
            \begin{array}{c}
                \Delta^i_j \in \Pi^i \qquad \Lambda^i \cup \{i : l \then \pos{a}m\}\\ \cline{1-1} 
                \Delta^i_j \cup \{\neg l\}\ |\ \Delta^i_j \cup \{l, \pos{a}m\} 
            \end{array}
            &
            \mbox{(NEG)}
        \end{array}
            }
            &
            \ensuremath{\begin{array}{cc}
            \begin{array}{c}
                \Delta^i_j \in \Pi^i \qquad \Theta^i \cup \{i : l \then \nec{a}m\}\\ \cline{1-1}
                \Delta^i_j \cup \{\neg l\}\ |\ \Delta^i_j \cup \{l, \nec{a}m\} 
            \end{array}
            &
            \mbox{(POS)}
        \end{array}
            }
        \end{tabular}
\end{tabular}
}
\end{center}
\end{framed}
\end{figure*}

In addition to the inference rules for construction of sets, we also have two
rules to eliminate sets that should not be part of the proof: 
\begin{itemize}
    \item[] (ELIM1): Eliminate sets containing both $l$ and $\neg l$ for some $l$.
    \item[] (ELIM2): If, at some modal level, we have that every set satisfy
        $\neg m_0 \lor \neg m_1 \lor \ldots \lor \neg m_r$, eliminate the
        sets at the previous level that contains $\pos{a} m_0, \nec{a} m_1, \ldots, \nec{a}
        m_r$.
\end{itemize}

A tableau proof, or just tableau, that has all the sets at the initial modal
level eliminated is said do be \emph{closed}. A tableau is \emph{open} if it's
not closed. A set of clauses that has an open tableau is said to be
\ckn-satisfiable.

\section{Correctness Results}
\label{sec:correctnessresults}

\subsection{Soundness}
\label{sec:sound}

A calculus is sound when it does not prove anything that it shouldn't. A tableau
calculus for a logic $L$ containing only sound rules, with respect to $L$, is
sound~\cite{gore2009clausal}. A sound rule has a satisfiable conclusion every
time its premisses are satisfiable.

\begin{lemma}{(PROP)} Let $\gamma^i = \cprop{k}{t}$ be a literal clause in
    \snf{ml} such that $\gamma^i \in \Gamma^i$, and $\Delta^i_j \in \Pi^i$ a
    literal set at the \ml-th modal level. If $\gamma^i$ and $\Delta^i_j$ are
    both satisfiable in \system{K}{n}{} then exists $1 \leq r \leq t$ such that
    $\Delta^i_j \cup \{l_r\}$ is also satisfiable in \system{K}{n}{}.
\end{lemma}
\begin{proof}
   Let $\Pi^i$ be the initial set of literal clauses of the \ml-th modal level,
   with $\gamma^i = \cprop{k}{t} \in \Gamma^i$, and consider $\Delta^i_j \in
   \Pi^i$ a literal set that, by hypothesis, also belongs to the \ml-th modal
   level. Supose that these hypothesis are satisfiable in \system{K}{n}{}. By
   the definition of satisfiability, there is a model $\Model = (W,
   \st_0, R_1, \ldots, R_n, \pi)$ and a world $\st \in W$ such that:
   \begin{itemize}
       \item $mdepth(\st) = \ml$ 
       \item \sat{\Model}{\st}{\bigvee_{k=1}^t l_k} and
       \item \sat{\Model}{\st}{d}, $\forall~d \in \Delta^i_j$
   \end{itemize}
   To prove that this rule is sound, it is sufficient to proof that, in an
   instancy of it, at least one of its denominators is satisfiable. In fact, we
   have that:
   \begin{itemize}
       \item \sat{\Model}{\st}{\bigvee_{k=1}^t l_k} $\iff$
           \sat{\Model}{\st}{l_1} ou $\ldots$ ou \sat{\Model}{\st}{l_t}, by
           Definition~\ref{relsat}
   \end{itemize}
   This means that exists at least one index $r$, with $1 \leq r \leq t$, such
   that \sat{\Model}{\st}{l_r}. Therefore, $\Delta^i_j \cup \{l_r\}$ is
   satisfiable in \system{K}{n}{}.
\end{proof}

\begin{lemma}{(NEG)} Let $\lambda^i = \cneg$ be an $a$-negative clause in \snf{ml}
    such that $\lambda^i \in \Lambda^i$, and $\Delta^i_j \in \Pi^i$ a literal
    set at the \ml-th modal level. If $\lambda^i$ and $\Delta^i_j$ are both
    satisfiable in \system{K}{n}{} then either $\Delta^i_j \cup
    \{\neg l\}$ or $\Delta^i_j \cup \{l, \pos{a}m\}$ is also satisfiable in
    \system{K}{n}{}.
\end{lemma}
\begin{proof}
   Let $\Lambda^i$ be the initial set of negative modal clauses at the \ml-th
   modal level, with $\lambda^i = \cneg \in \Lambda^i$, and consider $\Delta^i_j
   \in \Pi^i$ a literal set that, by hypothesis, also belongs to the \ml-th
   modal level. Supose that these hypothesis are all satisfiable in
   \system{K}{n}{}. By the definition of satisfiability, there exists a
   model $\Model = (W, \st_0, R_1, \ldots, R_n, \pi)$ and a world $\st \in W$
   such that:
   \begin{itemize}
       \item $mdepth(w) = \ml$ 
       \item \sat{\Model}{\st}{l \then\pos{a} m}
       \item \sat{\Model}{\st}{d}, $\forall~d \in \Delta^i_j$
   \end{itemize}
   To prove that this rule is sound, it is sufficient to proof that, in an
   instancy of it, one of the denominators is satisfiable. In fact, we have
   that:
   \begin{itemize}
       \item \sat{\Model}{\st}{l \then\pos{a} m} $\iff \sat{\Model}{\st}{\neg
           l}$ ou \sat{\Model}{\st}{\pos{a}m}, by Definition~\ref{relsat}
   \end{itemize}
   If $\sat{\Model}{\st}{\neg l}$ holds then $\Delta^i_j \cup \{\neg l\}$ is
   satisfiable. Otherwise, $\langle \Model, \st \rangle \not \models \neg l$ and
   \sat{\Model}{\st}{\pos{a}m} hold, therefore, $\Delta^i_j \cup \{l, \pos{a}
   m\}$ is satisfiable in \system{K}{n}{}, by Definition~\ref{relsat}. 
\end{proof}

\begin{lemma}{(POS)} Let $\theta^i = \cpos$ be an $a$-positive clause in \snf{ml}
    such that $\theta^i \in \Theta^i$, and $\Delta^i_j \in \Pi^i$ a literal
    set in the \ml-th modal level. If $\theta^i$ and $\Delta^i_j$ are both
    satisfiable in \system{K}{n}{} then at least one of $\Delta^i_j \cup
    \{\neg l\}$ or $\Delta^i_j \cup \{l, \pos{a}m\}$ is also satisfiable in
    \system{K}{n}{}.
\end{lemma}

Being the rule (POS) quite analog to the rule (NEG), we leave the proof of its
soundness omitted.

The proofs for elimination rules use the contrapositive.

\begin{lemma}{(ELIM1)} Let $\Delta^i_j \in \Pi^i$ be a literal set at the \ml-th
    modal level. If $\Delta^i_j$ is satisfiable in \system{K}{n}{} then
    $\Delta^i_j$ does not contain both $l$ and $\neg l$ for any $l \in
    \Delta^i_j$.
\end{lemma}
\begin{proof}
    Trivial by the definition of satisfiability.
\end{proof}

\begin{lemma}{(ELIM2)} Let $\Delta^i_k \in \Pi^i$ be a literal set at the \ml-th
    modal level such that $\{\pos{a} m_0, \nec{a} m_1, \ldots, \nec{a} m_r\}
    \subseteq \Delta^i_k$ for some $m_0, m_1, \ldots, m_r \in \Literals, r \geq 0$.
    If $\Delta^i_k$ is satisfiable in \system{K}{n}{} then exists
    $\Delta^{i+1}_j \in \Pi^{i+1}$ such that $\Delta^{i+1}_j \cup \{m_0, m_1,
    \ldots, m_r\}$ is also satisfiable.
\end{lemma}
\begin{proof}
    Let's assume that $\Delta^i_k \cup \{\pos{a}m_0\} \cup \{\nec{a}m_1, \ldots,
    \nec{a}m_r\} \in \Pi^i$ is satisfiable in \system{K}{n}{}, i.e., that exists
    a model $\Model = (W, \st_0, R_1, \ldots, R_n, \pi)$ and a world $\st \in
    W$, with $mdepth(\st) = \ml$, such that, by Definition~\ref{relsat}:
    \begin{itemize}
        \item
            \sat{\Model}{\st}{\Delta^i_k\cup\{\pos{a}m_0\}\cup\{\nec{a}m_1,\ldots,
            \nec{a}m_r\}}
    \end{itemize}
    And by definition of satisfiability in sets:
    \begin{itemize}
        \item[(1)]
            \sat{\Model}{\st}{d\land\pos{a}m_0\land\nec{a}
            m_1\land\ldots\land\nec{a} m_r, \forall~d\in\Delta^i_k}
    \end{itemize}
    We have, then, \sat{\Model}{\st}{\pos{a}m_0}. Therefore, it exists a world
    $\st'$, where $\st R_a \st'$ and $mdepth(\st')=\ml + 1$, where
    \sat{\Model}{\st'}{m_0}. As \sat{\Model}{\st}{\nec{a}m_k} holds for every $0
    \leq k \leq r$, and we already know that $\st R_a \st'$, then
    \sat{\Model}{\st'}{m_k} also holds, for every $0 \leq k \leq r$, by the
    definitions of satisfiability of the operators $\pos{a}$ and $\nec{a}$.

    Supose, by contradiction, that it doesn't exist $\Delta^{i+1}_j \in
    \Pi^{i+1}$, literal set at $(\ml+1)$-th modal level, such that
    $\Delta^{i+1}_j\cup\{m_0,m_1,\ldots,m_r\}$ is satisfiable in \Model. That
    means that every set $\Delta^{i+1}_j$ satisfies $\neg m_0 \lor \neg m_1 \lor
    \ldots \lor \neg m_r$. Then, again by definition of satisfiability of
    $\nec{a}$, we have that $\sat{\Model}{\st''}{\nec{a}(\neg m_0\lor\neg
    m_1\lor\ldots\lor\neg m_r)}$ for all $\st''$ with $mdepth(\st'')=i$. In
    particular, because $mdepth(\st) = i$, we have that:
    \begin{itemize}
        \item[(2)] $\sat{\Model}{\st}{\nec{a}(\neg m_0 \lor \neg m_1 \lor \ldots
            \lor \neg m_r)}$
    \end{itemize}
    From (1) and (2) and because $\nec{a}\formula\land\nec{a}\psi$ is
    semantically equivalent to $\nec{a}(\formula\land\psi)$, we obtain that:
    \begin{itemize}
        \item[(3)] $\sat{\Model}{\st}{\nec{a}((\neg m_0 \lor \neg m_1 \lor
                \ldots \lor \neg m_r)\land m_1 \land \ldots \land m_r) \land
            \pos{a} m_0}$ 
    \end{itemize}
    By resolution's principle, we have $\sat{\Model}{\st}{\nec{a}\neg m_0 \land
    \pos{a}m_0}$. Finally, by satisfiability definition of $\pos{a}$ and
    $\nec{a}$, we obtain \sat{\Model}{\st}{\textbf{false}}, a contradiction.
    Thus, the suposition of not existence of such $\Delta^{i+1}_j$ is untrue.
    Therefore, the conclusion of the rule is satisfiable.
\end{proof}

\subsection{Completeness}
\label{sec:complete}
%It is complete if for all finite sets F of formulae,
%if F is C L -satisfiable then F is L -satisfiable.

A calculus is complete when it proves everything that it should. We show that
\ckn~is complete, by showing that for all sets of clauses $C$ that have an open
tableau, that is, for all sets that is \ckn-satisfiable, $C$ must also be
\system{K}{n}{}-satisfiable.

\begin{definition}
\label{def:saturated} 
    We say that a \ckn~tableau proof is \emph{saturated} if all apropriate rule
    applications have been made. More precisely, an open tableau is saturated
    if:
    \begin{enumerate}
        \item All literal clauses, $a$-positive and $a$-negative clauses have
            been treated. That is, the (PROP), (NEG) and (POS) rules have been
            aplied to all clauses in $C$, appropriately.
        \item As a result of applying the rule (ELIM1), there is no set
            containgin both $l$ and $\neg l$, for any $l$, at any modal level.
        \item At all modal levels $i$, if there is a set $\Delta^i$ that contains a
            diamond, it must be a set at modal level $i+1$ that satisfies the
            literal in the scope of the diamond and all the literals in the
            scope of the boxes that are in $\Delta^i$. Otherwise, this set would
            have been eliminated by the application of (ELIM2).
    \end{enumerate}
\end{definition}

If $\cal{T}$ is a saturated \ckn~open tableau for $C$, we can extract a model
that satisfies $C$. This is showed in Theorem~\ref{completeness}.
%If, for instance, $\cprop{r}{k} \in C$, we know that exists a set of literals
%$\Delta^i$ at modal level \ml~such that $l_k \in \Delta^i$ for some $1 \leq k
%\leq r$, by item one. Also by item one, we know that if $\cneg \in C$, it has to
%exist a set $\Delta^i$ that satisfies either $\neg l$ or $l \land \pos{a} m$.
%This last statement is analogous for an $a$-positive clause. Finaly, if, at any
%modal level \ml, we have a set $\Delta^i$ that contains $\pos{a} m_0$ and
%$\nec{a} m_1, \ldots, \nec{a} m_r$, with $r \geq 0$, we will also have a set at
%modal level $i+1$ that satisfies $m_0, m_1, \ldots, m_r$, by item three. 

We now specify a systematic construction procedure that, when followed,
generates either a closed or a saturated tableau for any set of clauses
$C$.

\begin{enumerate}
    \item Initially, apply the rule (PROP) to all literal clauses at all modal
        levels, eliminating the sets where inconsistency is inserted. If all the
        sets of the first modal level are eliminated, return the closed tableau.
    \item Next, apply the rules (NEG) and (POS) to all modal clauses at all modal
        levels, always eliminating the inconsistent sets. Again, if the first
        modal level is empty, return the closed tableau.
    \item Finally, from the maximum modal level downto the first one do as
        follows: For each diamond not yet satisfied, search in the next modal
        level for a set to satisfy the literals in the scope of this diamond and
        in the scope of every box belonging to the same set. The search fails
        when the rule (ELIM2) can be applied to the set holding this diamond.
        If all sets at the first modal level are eliminated, return the closed
        tableau. Otherwise, return the saturated tableau generated.
\end{enumerate}

There is some flexibility entailed on the procedure described in the sense that
it was left open which untreated clause to choose at the first and second
phases, and also which unsatisfied diamonds to deal with at the last phase. But
it can be shown that, no matter how the choice is made, the process must
terminate.

\begin{lemma}
    The systematic construction procedure of a saturated tableau described above
    terminates for any set of clauses $C$.
\end{lemma}
\begin{proof}
\end{proof}

\begin{theorem}
\label{completeness}
    Let $C$ be a set of clauses in \snf{K}. If $C$ is \ckn-satisfiable then it
    is \system{K}{n}{}-satisfiable.
\end{theorem}
\begin{proof}
    Let $C$ be a \ckn-satisfiable set of clauses, thus, following the procedure
    described above, we are able to generate a saturated open tableau
    $\mathcal{T}$ for $C$. To proof that $C$ is also satisfiable in
    \system{K}{n}{}, we need to build a model $\Model = (\St, \st_0, R_1, \ldots, R_n, \pi)$ that satisfies $C$. We do this
    using the information in $\mathcal{T}$.

    If $\iota$ is the largest modal level in $\mathcal{T}$, take $W = \Pi^1 \cup
    \ldots \cup \Pi^\iota$ and $\pi(\Delta^i_j, p) = $ \textbf{true} if $p \in
    \Delta^i_j$ and $\pi(\Delta^i_j, p) =$ \textbf{false}, otherwise.

    From the systematic procedure, if $\Delta^{i+1}_k$ is the return of the
    search for a set that satisfies the diamond $\pos{a}$ in $\Delta^i_j$, take
    $R_{\agent} \cup \{(\Delta^i_j, \Delta^{i+1}_k)\}$.

    Finally, any set $\Delta^1_j \in \Pi^1$ can be taken as $\st_0$.

    Now, to proof that \Model~satisfies $C$,
\end{proof}

\section{Example}
\label{sec:example}

$\nec{}(p \then q) \then (\nec{} p \then \nec{} q)$

\input{tex/07.relatedwork.tex}

\bibliographystyle{plain}
\bibliography{bib/lsfa2017,bib/logic,bib/proof}

\end{document}
