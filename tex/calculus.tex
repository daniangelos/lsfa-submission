\section{Calculus}
\label{sec:calculus}

The proposed calculus, denoted by \ckn, comprehends a set of inference rules to
deal with both propositional and modal reasoning. Before presenting these rules,
it is necessary to define the notation that is used in them, in order to
establish the formality needed.

\begin{definition}
\label{def_conjuntos}
Let $\gamma^i, \lambda^i$ e $\theta^i$ be representations of 
literal, $\agent$-negative and $\agent$-positive clauses, respectively,
as follows:
\begin{enumerate}
\item $\gamma^i \stackrel{def} i : \bigvee^t_{k=1} l_k$
    \vspace{0.4em}
\item $\lambda^i \stackrel{def} i : l \then \pos{a}m$
    \vspace{0.4em}
\item $\theta^i \stackrel{def} i : l \then \nec{a}m$
\end{enumerate}

Also, let $\Gamma^i, \Lambda^i$ e $\Theta^i$ be the initial set of the  
respective clauses at the \ml-th modal level:
\begin{enumerate}
   \item $\Gamma^i \stackrel{def} \{\gamma_1^i, \ldots, \gamma_k^i\}$, $k \in \mathbb{N}$
    \vspace{0.4em}
   \item $\Lambda^i \stackrel{def} \{\lambda_1^i, \ldots, \lambda_j^i\}$, $j \in \mathbb{N}$
    \vspace{0.4em}
   \item $\Theta^i \stackrel{def} \{\theta_1^i, \ldots, \theta_t^i\}$, $t \in \mathbb{N}$
\end{enumerate}

Let $\iota$ be the maximum modal level of a clause in any of the sets defined
above and consider then that the initial set of clauses is represented by:
\begin{enumerate}
    \item $C = (\bigcup_{i=1}^{\iota} \Gamma^i) \cup (\bigcup_{i=1}^{\iota} \Lambda^i)
        \cup (\bigcup_{i=1}^{\iota} \Theta^i)$ 
\end{enumerate}

Finally, consider $i: \{d_1, \ldots, d_t\} = \Delta^i$ as a set of literals, which
occur either on propositional or modal scope, that belongs to the modal
level $i$ with $\Pi^i = \{\Delta^i_1, \ldots, \Delta^i_k\}$, for some $k \in
\mathbb{N}$, denoting all the literals sets at the same modal level. Observe
that if $d \in \Delta^i_j$ then $d \in \mathcal{L}$ or $d$ is of form
$\pos{a} l$ or $\nec{a} l$ (modal literal), for some $l \in \mathcal{L}$.  
\end{definition}

Initiallly, consider that $\Pi^\ml$ carries an empty set $\Delta_1^\ml$, for all
modal levels. We now present the inference rules that are aplied to theses sets
of literals.

The first inference rule, (PROP), related to literal clauses, has as numerator
all literal sets $\Delta^i_j$ that are already satisfied at modal level \ml, and
considers a clause $\gamma^\ml = \cprop{t}{k}$ of $\Gamma^\ml$ that has not been
treated. The application of (PROP) for the clause $\gamma^\ml$ results in the
branching of each set of the numerator to consider every literal in the
disjunction.  The intuition for this rule is that if $\Delta^i_j$ and
$\gamma^\ml \in \Gamma^\ml$ are both satisfiable, then $\Delta^i_j \cup \{l_r\}$
has to be satisfiable for some $1 \leq r \leq t$, therefore, at least one of the
denominators is satisfiable.

The rules (NEG) and (POS) refer to modal clauses. Their meaning are pretty
similar, even though they deal with different types of modal clauses, the
difference is only the semantics of the operators. Ergo, the following
description refers to the rule (NEG), being the one for the rule (POS)
analogous.

This rule has as numerator all literal sets $\Delta^i_j$ that are already
satisfied at modal level \ml, and considers the $a$-negative modal clauses
$\lambda^i = \cneg$ of $\Lambda^\ml$ at the same level. The application of (NEG)
results in the duplication of $\Delta^i_j$ to consider both sides of the
implication, as the implication is, actually, a disjuction between the head of the
clause in negative form and the modal literal that occurs as the consequent
of it ($\neg l \lor\pos{a} m$). By hypothesis, $\Delta^i_j$ and $\lambda^i$ are
both satisfiable, so at least one of the denominators, $\Delta^i_j \cup \{\neg
l\}$ or $\Delta^i_j \cup \{l,\pos{a}m\}$ must also be satisfiable.

\begin{figure*}
\begin{framed}
\begin{center}
    \small{\
    %\begin{tabular}{c}
        
        \begin{tabular}{cc}
        \ensuremath{\begin{array}{cc}
            \begin{array}{l}
                i : \bigvee^t_{k=1} l_k  \qquad \Delta^i_j \in \Pi^i\\ \cline{1-1}
                \Delta^i_j \cup \{l_1\}\ |\ \ldots\ |\ \Delta^i_j \cup \{l_t\}
            \end{array}
            &
            \mbox{(PROP)}
        \end{array}
    %\caption{Regra de inferência do tableaux para cláusulas
    %proposicionais}
            }

            %&

        %\ensuremath{\begin{array}{cc}
            %\begin{array}{l}
                %\Pi^i \cup \Delta^i \\ 
                %\Delta^i \cup \{p, \neg p\} \\ \cline{1-1}
                %\begin{array}{c}\Pi^i\end{array}
            %\end{array}
            %&
            %\mbox{(ELIM1)}
        %\end{array}
        %}
        %\end{tabular}

\vspace{1.5em}
\\

\begin{tabular}{cc}
    \ensuremath{\begin{array}{cc}
            \begin{array}{c}
                i : l \then \pos{a}m \qquad \Delta^i_j \in \Pi^i \\ \cline{1-1} 
                \Delta^i_j \cup \{\neg l\}\ |\ \Delta^i_j \cup \{l, \pos{a}m\} 
            \end{array}
            &
            \mbox{(NEG)}
        \end{array}
            }
            &
            \ensuremath{\begin{array}{cc}
            \begin{array}{c}
                i : l \then \nec{a}m \qquad \Delta^i_j \in \Pi^i \\ \cline{1-1} 
                \Delta^i_j \cup \{\neg l\}\ |\ \Delta^i_j \cup \{l, \nec{a}m\} 
            \end{array}
            &
            \mbox{(POS)}
        \end{array}
            }
        \end{tabular}
    \end{tabular}
}
\end{center}
\end{framed}
\end{figure*}

In addition to the inference rules for constructing a proof by adding literals
when branching sets, we also have two rules to eliminate sets that should not be
a part of the proof: 
\begin{itemize}
    \item[] (ELIM1): Eliminate sets containing both $l$ and $\neg l$ for some $l$.
    \item[] (ELIM2): If, at some modal level, we have that every set satisfy
        $\neg m_0 \lor \neg m_1 \lor \ldots \lor \neg m_r$, eliminate the
        sets at the previous level that contains $\pos{a} m_0, \nec{a} m_1, \ldots, \nec{a}
        m_r$.
\end{itemize}

