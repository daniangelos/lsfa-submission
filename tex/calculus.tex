\section{Calculus}
\label{sec:calculus}

The proposed calculus, denoted by \ckn, comprehends a set of inference rules to
deal with both propositional and modal reasoning. Before presenting these rules,
it is necessary to define the notation that is used in them, in order to
establish the formality needed.

\begin{definition}
\label{def_conjuntos}
Let $\gamma^i, \lambda^i$ e $\theta^i$ be representations of 
literal, $\agent$-negative and $\agent$-positive clauses, respectively,
as follows:
\begin{enumerate}
\item $\gamma^i := i : \bigvee^t_{k=1} l_k$
    \vspace{0.4em}
\item $\lambda^i := i : l \then \pos{a}m$
    \vspace{0.4em}
\item $\theta^i := i : l \then \nec{a}m$
\end{enumerate}

Also, let $\Gamma^i, \Lambda^i$ e $\Theta^i$ be the initial set of the  
respective clauses at the modal level \ml:
\begin{enumerate}
   \item $\Gamma^i := \{\gamma_1^i, \ldots, \gamma_p^i\}$, $p \in \mathbb{N}$
    \vspace{0.4em}
   \item $\Lambda^i := \{\lambda_1^i, \ldots, \lambda_q^i\}$, $q \in \mathbb{N}$
    \vspace{0.4em}
   \item $\Theta^i := \{\theta_1^i, \ldots, \theta_r^i\}$, $r \in \mathbb{N}$
\end{enumerate}

Let $\iota$ be the maximum modal level of a clause in any of the sets defined
above and consider then that the initial set of clauses is represented by:
\begin{enumerate}
    \item $C = (\bigcup_{i=1}^{\iota} \Gamma^i) \cup (\bigcup_{i=1}^{\iota} \Lambda^i)
        \cup (\bigcup_{i=1}^{\iota} \Theta^i)$ 
\end{enumerate}

Finally, consider $\Delta^i$ a set of literals, either propositional or modal,
that belongs to the modal level $i$ with $\Pi^i = \{\Delta^i_1, \ldots,
\Delta^i_k\}$, for some $k \in \mathbb{N}$. Observe that if $d \in \Delta^i_j$
then $d \in \mathcal{L}$ or $d$ is of form $\pos{a} l$ or $\nec{a} l$
(modal literal), for some $l \in \mathcal{L}$.
\end{definition}

Initiallly, consider that $\Pi^\ml$ carries an empty set $\Delta_1^\ml$, for all
modal levels. We now present the inference rules that are aplied to theses sets
of literals.

The first inference rule, (PROP), related to literal clauses, has as numerator
all literal sets that are already satisfied at modal level \ml, and considers a
clause $\gamma^\ml = \cprop{t}{k}$ of $\Gamma^\ml$ that has not been treated. The
application of (PROP) for the clause $\gamma^\ml$ results in the branching of
each set of the numerator to consider every literal in the disjunction.  The
intuition for this rule is that if $\Delta^i_j$ and $\gamma^\ml \in \Gamma^\ml$
are both satisfiable, then $\Delta^i_j \cup \{l_r\}$ has to be satisfiable for
some $1 \leq r \leq t$, therefore, at least one of the denominators is
satisfiable.

The rules (NEG) and (POS) refer to modal clauses. Their meaning are pretty
similar, even though they deal with different types of modal clauses, the
difference is only the semantics of the operators. Ergo, the following
description refers to the rule (NEG), being the one for the rule (POS)
analogous.

This rule has as numerator all literal sets $\Delta^i_j$ that are already
satisfied at modal level \ml, and considers the $a$-negative modal clauses
$\lambda^i = \cneg$ of $\Lambda^\ml$ at the same level. The application of (NEG)
results in the duplication of $\Delta^i_j$ to consider both sides of the
implication, as the implication is, actually, a disjuction between the head of the
clause in negative form and the modal literal that occurs as the consequent
of it ($\neg l \lor\pos{a} m$). By hypothesis, $\Delta^i_j$ and $\lambda^i$ are
both satisfiable, so at least one of the denominators, $\Delta^i_j \cup \{\neg
l\}$ or $\Delta^i_j \cup \{l,\pos{a}m\}$ must also be satisfiable.

The last calculus' rule, (EXP), has as numerator all sets $\Delta^i_k \in
\Pi^i$, at the \ml-th modal level, such that $\{\pos{a}m_0, \nec{a}m_1, \ldots,
\nec{a}m_r\} \subseteq \Delta^i_k$ for some $m_0, m_1, \ldots, m_r \in
\Literals, r \geq 0$. That means that this set contains at least one diamond and
an arbitrary number of boxes.  This rule expresses that if $\Delta^i_k$ is
satisfiable then exists at least one set $\Delta^{i+1}_j \in \Pi^{i+1}$ that
satisfies $\{m_0, m_1, \ldots, m_r\}$, that is, each literal in a diamond must
be satisfied at the next modal level along with all the literals in the boxes in
the same set.

\begin{figure*}
\begin{framed}
\begin{center}
    \small{\
    \begin{tabular}{c}
        
        \ensuremath{\begin{array}{cc}
            \begin{array}{c}
                \Delta^i_j \in \Pi^i \qquad \Gamma^i \cup \{i : \bigvee^t_{k=1} l_k\} \\ \cline{1-1}
                \Delta^i_j \cup \{l_1\}\ |\ \ldots\ |\ \Delta^i_j \cup \{l_t\}
            \end{array}
            &
            \mbox{(PROP)}
        \end{array}
    %\caption{Regra de inferência do tableaux para cláusulas
    %proposicionais}
            }
\vspace{1.5em}
\\

\begin{tabular}{cc}
    \ensuremath{\begin{array}{cc}
            \begin{array}{c}
                \Delta^i_j \in \Pi^i \qquad \Lambda^i \cup \{i : l \then \pos{a}m\}\\ \cline{1-1} 
                \Delta^i_j \cup \{\neg l\}\ |\ \Delta^i_j \cup \{l, \pos{a}m\} 
            \end{array}
            &
            \mbox{(NEG)}
        \end{array}
            }
            &
            \ensuremath{\begin{array}{cc}
            \begin{array}{c}
                \Delta^i_j \in \Pi^i \qquad \Theta^i \cup \{i : l \then \nec{a}m\}\\ \cline{1-1}
                \Delta^i_j \cup \{\neg l\}\ |\ \Delta^i_j \cup \{l, \nec{a}m\} 
            \end{array}
            &
            \mbox{(POS)}
        \end{array}
            }
        \end{tabular}
\vspace{1.5em}
        \\
        \ensuremath{\begin{array}{cc}
                \begin{array}{c}
                    \Delta^i_k \cup \{\pos{a} m_0\} \cup \{\nec{a}m_1, \ldots, \nec{a}m_r\} \in \Pi^i \\ \cline{1-1} 
                    \Delta^{i+1}_j \cup \{m_0, m_1, \ldots, m_r\}
                \end{array}
                &
                \mbox{for some }\Delta^{i+1}_j \in\Pi^{i+1} \quad \mbox{(EXP)}
                \end{array}
            }
    \end{tabular}
}
\end{center}
\end{framed}
\end{figure*}

When branching a set of literals results in an inconsistency, i.e., when we
generate a set that contains both $d$ and $\neg d$ for some $d$, this set is
eliminated from the proof. Furthermore, if we are not able to satisfy a diamond
of a specific set, along with all the boxes in the same set, we also eliminate
this set of the proof. A proof that has all the sets at a specific modal level
eliminated is called a closed tableau. A closed tableau means that the initial set
of clauses $C$ was unsatisfiable while an open tableu gives us enough
semantic information to construct a model that witnesses the satisfiability of
these clauses.

