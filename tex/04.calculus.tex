The proposed calculus, denoted by \ckn, comprehends a set of inference rules to
deal with both propositional and modal reasoning. Before presenting the
inference rules, we need to define some more notation. Let $\set{C}$ be a set of
clauses and $i \in \Nat$. Consider that $\gamma^i$ names a literal clause in
$\set{C}$ occurring at the modal level $i$, that is, a clause of the form $i:
\bigvee^r_{b=1} l_b$; $\lambda^i$ names a negative $\agent$-clause in $\set{C}$,
that is, a clause of the form $i : l \then \pos{\agent}m$, for all $\agent \in
\Agents$; and $\theta^i$ a positive $\agent$-clause in $\set{C}$, that is, a
clause of the form $i : l \then \nec{a}m$, for all $\agent \in \Agents$. We use
the capital Greek letters $\Gamma^i, \Lambda^i$ and $\Theta^i$ to denote the
respective sets of clauses. We denote by $\Delta^i$ a set of literals and modal
literals occurring at the modal level $i$. Initially, $\Delta^i =\emptyset$.
Finally, we denote by $\Pi^i$ all sets of literals occurring at the modal
level $i$.

We now present the inference rules that are aplied to theses sets of literals. The inference rules try to build sets of literals for each modal level, starting with $\Pi^i = \{\emptyset\}$ (because $\Delta^i = \emptyset$ initially). 

The first inference rule, (PROP), is applied to literal clauses. It takes as premisse the sets of literals already built and expands these sets with literals occurring in clauses in $\Gamma_i$, as shown in the conclusion of the rule. The
intuition for this rule is that if $\Delta^i_j$ and $\gamma^\ml \in \Gamma^\ml$
are both satisfiable, then $\Delta^i_j \cup \{l_r\}$ has to be satisfiable for
some $1 \leq r \leq t$, therefore, at least one of the conclusions is
satisfiable.

The rules (NEG) and (POS) are applied to modal clauses. For (NEG), the premise
is applied to every set of literals $\Delta^i_j$ already built and to a negative
$\agent$-clause. As negative $a$-clauses can be seen as disjunctions, the
conclusion is a branching, where the negation of the left-hand side is added to
one of the sets and both the left-hand side and the right-hand side is added to
the other set. The inference rule (POS) is similar.

\begin{figure*}
\begin{framed}
\begin{center}
    \small{\
    \begin{tabular}{c}
        
        \ensuremath{\begin{array}{cc}
            \begin{array}{c}
                \Delta^i_j \in \Pi^i \qquad \Gamma^i \cup \{i : \bigvee^t_{k=1} l_k\} \\ \cline{1-1}
                \Delta^i_j \cup \{l_1\}\ |\ \ldots\ |\ \Delta^i_j \cup \{l_t\}
            \end{array}
            &
            \mbox{(PROP)}
        \end{array}
    %\caption{Regra de inferência do tableaux para cláusulas
    %proposicionais}
            }
\vspace{1.5em}
\\

\begin{tabular}{cc}
    \ensuremath{\begin{array}{cc}
            \begin{array}{c}
                \Delta^i_j \in \Pi^i \qquad \Lambda^i \cup \{i : l \then \pos{a}m\}\\ \cline{1-1} 
                \Delta^i_j \cup \{\neg l\}\ |\ \Delta^i_j \cup \{l, \pos{a}m\} 
            \end{array}
            &
            \mbox{(NEG)}
        \end{array}
            }
            &
            \ensuremath{\begin{array}{cc}
            \begin{array}{c}
                \Delta^i_j \in \Pi^i \qquad \Theta^i \cup \{i : l \then \nec{a}m\}\\ \cline{1-1}
                \Delta^i_j \cup \{\neg l\}\ |\ \Delta^i_j \cup \{l, \nec{a}m\} 
            \end{array}
            &
            \mbox{(POS)}
        \end{array}
            }
        \end{tabular}
\end{tabular}
}
\end{center}
\end{framed}
\end{figure*}

In addition to the inference rules for construction of sets, we also have two
rules to eliminate sets that should not be part of the proof: 
\begin{itemize}
    \item[] (ELIM1): Eliminate sets containing both $l$ and $\neg l$ for some $l$.
    \item[] (ELIM2): If, at some modal level, we have that every set satisfy
        $\neg m_0 \lor \neg m_1 \lor \ldots \lor \neg m_r$, eliminate the
        sets at the previous level that contains $\pos{a} m_0, \nec{a} m_1, \ldots, \nec{a}
        m_r$.
\end{itemize}

A tableau proof, or just tableau, that has all the sets at the first modal
level eliminated is said do be \emph{closed}. A tableau is \emph{open} if it's
not closed. A set of clauses that has an open tableau is said to be
\ckn-satisfiable.
