\section{Language}
\label{sec:language}

The Modal Language \system{K}{n}{} is equivalent to its set of \emph{well-formed
formulae}, denoted \wff, which is constructed from an enumerable set of
\emph{propositional symbols} $\mathcal{P} = \{p, q, r, \ldots\}$, the negation
symbol $\neg$, the disjunction symbol $\lor$ and the modal connectives
$\nec{a}$, that express the notion of necessity, for each index (or agent) $a$
in a finite, fixed set $\mathcal{A} = \{1, \ldots, n\}, n \in \mathbb{N}$.

The propositional symbols combined with the logic operators are arranged to form
sentences (parentheses can be used to avoid ambiguity also). Therefore, the set
of \wff~is recursively defined as showed in Definition~\ref{def:wff}.

\begin{definition}
\label{def:wff}
    The set of well-formed formulae, \wff, is the least set such that:
    \begin{enumerate}
        \item $\mathcal{P} \subset \wff$
            \vspace{.2ex}
        \item if $\varphi, \psi \in \wff$, then so are $\neg \varphi, (\varphi
            \lor \psi)$ and $\nec{a} \varphi$, for each $a \in \Agents$
    \end{enumerate}
\end{definition}

As one might know, other logic operators may be introduced as abbreviation to
formulae constructed using the operators defined as essential. In particular,
this paper considers the abbreviations:
$\varphi \wedge \psi \stackrel{def} \neg(\neg \varphi \lor \neg \psi)$
(conjuction),
$\varphi \then \psi \stackrel{def} \neg \varphi \lor \psi$ (implication),
$\varphi \iff \psi \stackrel{def} (\varphi \then \psi) \land (\psi \then
\varphi)$ (equivalence),
$\pos{a} \varphi \stackrel{def} \neg \nec{a} \neg \varphi$ (possibility),
$\textbf{false} \stackrel{def} \varphi \wedge \neg \varphi$ (\emph{falsum}),
$ \textbf{true} \stackrel{def} \neg \textbf{false}$ (\emph{verum}).
And the precedence of these operators is in the order: 
\{$\neg, \nec{a}, \pos{a}$\},
\{$\wedge$\},
\{$\lor$\},
\{$\then$\},
\{$\iff$\}.

For simplicity, we often refer to $\nec{a} \formula$ as \emph{box} \formula~and to
$\pos{a} \formula$ as \emph{diamond} \formula.

Logics that involve $n$ agents in the modal logic, with $n \in \Nat$, are know
as Multimodal Logics. When $n = 1$, we tend to omit the index in the modal
operators, i.e., we just write $\nec{}$ and $\pos{}$.


The maximal number of modal operators in a formula is defined as its \emph{modal
depth} and denoted $mdepth$. The maximal number of modal operators in which
scope the formula occurs is defined as the \emph{modal level} of that formula,
and it is denoted $ml$.
For instance, in $\nec{a}\pos{a} p$, $mdepth(p) = 0$ and $ml(p) = 2$.

\subsection{Semantics}
\label{sec:semantics}

The semantics of \system{K}{n}{} is presented in terms of models based on Kripke
structures.

\begin{definition}
    A Kripke model for the set of propositional symbols \Prop~and the agents
    \Agents~is given by the tuple $\Model = (W, \st_0, R_1, \ldots, R_n, \pi)$,
    where $W$ is a non-empty set of possible worlds with a distinguinshed world
    $\st_0$, the root of \Model, each $R_a, a \in \Agents$, is a binary relation
    on $W$, and $\pi: W \times \Prop \longrightarrow \{false, true\}$ is the
    valuation function that associates to each world $w \in W$ a
    truth-assignment to propositional symbols.
\end{definition}

From the definition of a Kripke model, one can define the satisfiability and
validity of a formula in \system{K}{n}{}.

\begin{definition}
\label{relsat}
    Let $\Model = (W, \st_0, R_1, \ldots, R_n, \pi)$ be a Kripke model for
    \Prop~and \Agents, and consider $\st \in W, p \in \Prop$ and $\formula,
    \psi \in \wff$. The \emph{satisfiability relation}, denoted by 
    \sat{\Model}{\st}{\formula}, among the world \st~and a formula \formula~in
    the model \Model,
    is inductively defined by:
    \begin{enumerate}
        \item \sat{\Model}{\st}{p} if, and only if, $\pi(\st, p) = \textbf{true}$ 
        \item \sat{\Model}{\st}{\neg\formula} if, and only if, $
            \langle \Model, \st \rangle \not \models \formula$
        \item \sat{\Model}{\st}{\formula\lor\psi} if, and only if,
            \sat{\Model}{\st}{\formula} or \sat{\Model}{\st}{\psi}
        \item \sat{\Model}{\st}{\nec{a} \formula} if, and only if, $\forall~t\in
            W$, with $a \in \Agents$, $(\st, t) \in R_a$ implies
            \sat{\Model}{t}{\formula}
    \end{enumerate}
\end{definition}

A set of formulae $\mathcal{F} = \{\formula_1, \ldots, \formula_r\}, r \in \Nat$, is
satisfiable in a world \st~if, and only if, each of its formulae is satisfiable
in this world, i.e., \sat{\Model}{\st}{\mathcal{F}}
$\iff$ \sat{\Model}{\st}{\formula_1\land\ldots\land\formula_r}.

\begin{definition} % SATISFIABILITY
\label{satisfiability}
    A formula $\formula \in \wff$ is said to be \emph{satisfiable} if exists a
    Kripke model \Model~such that \sat{\Model}{\st_0}{\formula}.
\end{definition}

\begin{definition} % VALIDITY
    A formula $\formula \in \wff$ is said to be \emph{valid} if for all Kripke
    model \Model, we have that \sat{\Model}{\st_0}{\formula}.
\end{definition}

When one is considering a set of formulae instead of a single one, both
definitions of satisfiability and validity holds similarly to the definition of
satisfiability relation. That means you have to check the satisfiability (or
validity) of each formula in the set.

The basic satisfiability problem in \system{K}{n}{}, then, summarizes to
establish if a given set of formulae $\mathcal{F}$ is satisfiable in \system{K}{n}{}.
To perform this evaluation, there are several proof procedures available, each
one designed with a specific purpose to achieve.

