\section{Correctness Results}
\label{sec:correctnessresults}

\subsection{Soundness}
\label{sec:sound}

As mentioned in the Section~\ref{sec:tableauxclausal}, to prove that \ckn~is
sound, we just need to prove that its rules are sound.

\begin{lemma}{(PROP)} Let $\gamma^i \in \cprop{k}{t}$ be a literal clause in
    \snf{ml} such that $\gamma^i \in \Gamma^i$, and $\Delta^i_j \in \Pi^i$ a
    literal set at the \ml-th modal level. If $\gamma^i$ and $\Delta^i_j$ are
    both satisfiable in \system{K}{n}{} then exists $1 \leq r \leq t$ such that
    $\Delta^i_j \cup \{l_r\}$ is also satisfiable in \system{K}{n}{}.
\end{lemma}
\begin{proof}
   Let $\Pi^i$ be the initial set of literal clauses of the \ml-th modal level,
   with $\gamma^i = \cprop{k}{t} \in \Gamma^i$, and consider $\Delta^i_j \in
   \Pi^i$ a literal set that, by hypothesis, also belongs to the \ml-th modal
   level. Supose that these hypothesis are satisfiable in \system{K}{n}{}. By
   Definition~\ref{satisfiability}, it is know that exists a model $\Model = (W,
   \st_0, R_1, \ldots, R_n, \pi)$ and a world $\st \in W$ such that:
   \begin{itemize}
       \item $mdepth(\st) = \ml$ 
       \item \sat{\Model}{\st}{\bigvee_{k=1}^t l_k} and
       \item \sat{\Model}{\st}{d}, $\forall~d \in \Delta^i_j$
   \end{itemize}
   To prove that this rule is sound, it is sufficient to show that, in an
   instancy of it, at least one of its denominators is satisfiable. In fact, we
   have that:
   \begin{itemize}
       \item \sat{\Model}{\st}{\bigvee_{k=1}^t l_k} $\iff$
           \sat{\Model}{\st}{l_1} ou $\ldots$ ou \sat{\Model}{\st}{l_t}, by
           Definition~\ref{relsat}
   \end{itemize}
   This means that exists at least one index $r$, with $1 \leq r \leq t$, such
   that \sat{\Model}{\st}{l_r}. Therefore, $\Delta^i_j \cup \{l_r\}$ is
   satisfiable in \system{K}{n}{}.
\end{proof}

\begin{lemma}{(NEG)} Let $\lambda^i = \cneg$ be an $a$-negative clause in \snf{ml}
    such that $\lambda^i \in \Lambda^i$, and $\Delta^i_j \in \Pi^i$ a literal
    set in the \ml-th modal level. If $\lambda^i$ and $\Delta^i_j$ are both
    satisfiable in \system{K}{n}{} then at least one of $\Delta^i_j \cup
    \{\neg l\}$ or $\Delta^i_j \cup \{l, \pos{a}m\}$ is also satisfiable in
    \system{K}{n}{}.
\end{lemma}
\begin{proof}
   Let $\Lambda^i$ be the initial set of negative modal clauses at the \ml-th
   modal level, with $\lambda^i = \cneg \in \Lambda^i$, and consider $\Delta^i_j
   \in \Pi^i$ a literal set that, by hypothesis, also belongs to the \ml-th
   modal level. Supose that these hypothesis are all satisfiable in
   \system{K}{n}{}. By Definition~\ref{satisfiability}, it is know that exists a
   model $\Model = (W, \st_0, R_1, \ldots, R_n, \pi)$ and a world $\st \in W$
   such that:
   \begin{itemize}
       \item $mdepth(w) = \ml$ 
       \item \sat{\Model}{\st}{l \then\pos{a} m}
       \item \sat{\Model}{\st}{d}, $\forall~d \in \Delta^i_j$
   \end{itemize}
   To prove that this rule is sound, it is sufficient to show that, in an
   instancy of it, one of the denominators is satisfiable. In fact, we have
   that:
   \begin{itemize}
       \item \sat{\Model}{\st}{l \then\pos{a} m} $\iff \sat{\Model}{\st}{\neg
           l}$ ou \sat{\Model}{\st}{\pos{a}m}, by Definition~\ref{relsat}
   \end{itemize}
   If $\sat{\Model}{\st}{\neg l}$ holds then $\Delta^i_j \cup \{\neg l\}$ is
   satisfiable. Otherwise, $\langle \Model, \st \rangle \not \models \neg l$ and
   \sat{\Model}{\st}{\pos{a}m} hold, therefore, $\Delta^i_j \cup \{l, \pos{a}
   m\}$ is satisfiable in \system{K}{n}{}, by Definition~\ref{relsat}. 
\end{proof}

\begin{lemma}{(POS)} Let $\theta^i = \cpos$ be an $a$-positive clause in \snf{ml}
    such that $\theta^i \in \Theta^i$, and $\Delta^i_j \in \Pi^i$ a literal
    set in the \ml-th modal level. If $\theta^i$ and $\Delta^i_j$ are both
    satisfiable in \system{K}{n}{} then at least one of $\Delta^i_j \cup
    \{\neg l\}$ or $\Delta^i_j \cup \{l, \pos{a}m\}$ is also satisfiable in
    \system{K}{n}{}.
\end{lemma}

Being the rule (POS) quite analog to the rule (NEG), we leave the proof of its
soundness omitted.

\begin{lemma}{(EXP)} Let $\Delta^i_k \in \Pi^i$ be a literal set at the \ml-th
    modal level such that $\{\pos{a} m_0, \nec{a} m_1, \ldots, \nec{a} m_r\}
    \subseteq \Delta^i_k$ for some $m_0, m_1, \ldots, m_r \in \Literals, r \geq 0$.
    If $\Delta^i_k$ is satisfiable in \system{K}{n}{} then exists
    $\Delta^{i+1}_j \in \Pi^{i+1}$ such that $\Delta^{i+1}_j \cup \{m_0, m_1,
    \ldots, m_r\}$ is also satisfiable.
\end{lemma}
\begin{proof}
    Let's assume that $\Delta^i_k \cup \{\pos{a}m_0\} \cup \{\nec{a}m_1, \ldots,
    \nec{a}m_r\} \in \Pi^i$ is satisfiable in \system{K}{n}{}, i.e., that exists
    a model $\Model = (W, \st_0, R_1, \ldots, R_n, \pi)$ and a world $\st \in
    W$, with $mdepth(\st) = \ml$, such that, by Definition~\ref{satisfiability}:
    \begin{itemize}
        \item
            \sat{\Model}{\st}{\Delta^i_k\cup\{\pos{a}m_0\}\cup\{\nec{a}m_1,\ldots,
            \nec{a}m_r\}}
    \end{itemize}
    And by definition of satisfiability in sets:
    \begin{itemize}
        \item[(1)]
            \sat{\Model}{\st}{d\land\pos{a}m_0\land\nec{a}
            m_1\land\ldots\land\nec{a} m_r, \forall~d\in\Delta^i_k}
    \end{itemize}
    We have, then, \sat{\Model}{\st}{\pos{a}m_0}. Therefore, it exists a world
    $\st'$, where $\st R_a \st'$ and $mdepth(\st')=\ml + 1$, where
    \sat{\Model}{\st'}{m_0}. As \sat{\Model}{\st}{\nec{a}m_k} holds for every $0
    \leq k \leq r$, and we already know that $\st R_a \st'$, then
    \sat{\Model}{\st'}{m_k} also holds, for every $0 \leq k \leq r$, by the
    definitions of satisfiability of the operators $\pos{a}$ and $\nec{a}$.

    Supose, by contradiction, that it doesn't exist $\Delta^{i+1}_j \in
    \Pi^{i+1}$, literal set at $(\ml+1)$-th modal level, such that
    $\Delta^{i+1}_j\cup\{m_0,m_1,\ldots,m_r\}$ is satisfiable in \Model. That
    means that every set $\Delta^{i+1}_j$ satisfies $\neg m_0 \lor \neg m_1 \lor
    \ldots \lor \neg m_r$. Then, again by definition of satisfiability of
    $\nec{a}$, we have that $\sat{\Model}{\st''}{\nec{a}(\neg m_0\lor\neg
    m_1\lor\ldots\lor\neg m_r)}$ for all $\st''$ with $mdepth(\st'')=i$. In
    particular, because $mdepth(\st) = i$, we have that:
    \begin{itemize}
        \item[(2)] $\sat{\Model}{\st}{\nec{a}(\neg m_0 \lor \neg m_1 \lor \ldots
            \lor \neg m_r)}$
    \end{itemize}
    From (1) and (2) and because $\nec{a}\formula\land\nec{a}\psi$ is
    semantically equivalent to $\nec{a}(\formula\land\psi)$, we obtain that:
    \begin{itemize}
        \item[(3)] $\sat{\Model}{\st}{\nec{a}((\neg m_0 \lor \neg m_1 \lor
                \ldots \lor \neg m_r)\land m_1 \land \ldots \land m_r) \land
            \pos{a} m_0}$ 
    \end{itemize}
    By resolution's principle, we have $\sat{\Model}{\st}{\nec{a}\neg m_0 \land
    \pos{a}m_0}$. Finally, by satisfiability definition of $\pos{a}$ and
    $\nec{a}$, we obtain \sat{\Model}{\st}{\textbf{false}}, a contradiction.
    Thus, the suposition of not existence of such $\Delta^{i+1}_j$ is untrue.
    Therefore, the conclusion of the rule is satisfiable.
\end{proof}

\subsection{Completeness}
\label{sec:complete}
%It is complete if for all finite sets F of formulae,
%if F is C L -satisfiable then F is L -satisfiable.

We show that \ckn~is complete, by showing that for all set of clauses $C$ that
has an open tableau, that is, for all set that is \ckn-satisfiable, $C$ must
also be \system{K}{n}{}-satisfiable.

\begin{definition}
\label{def:saturated} 
    We say that a \ckn~tableau proof is \emph{saturated} if all apropriate rule
    applications have been made. More precisely, an open tableau is saturated
    if:
    \begin{enumerate}
        \item All literal clauses, $a$-positive and $a$-negative clauses have
            been treated. That is, the (PROP), (NEG) and (POS) rules have been
            aplied to all clauses in $C$, appropriately.
        \item In all modal levels $i$, if there is a set $\Delta^i$ that contains a
            diamond, it must be a set at modal level $i+1$ that satisfies the
            literal in the diamond and all the literals in boxes that are in
            $\Delta^i$. That means that the rule (EXP) was aplied correctly in
            all modal levels.
    \end{enumerate}
\end{definition}

If $\cal{T}$ is a saturated \ckn~tableau, we can extract several useful
information about it. If, for instance, $\cprop{r}{k} \in C$, we know that
exists a set of literals $\Delta^i$ at modal level \ml~such that $l_k \in
\Delta^i$ for some $1 \leq k \leq r$, by item 1. Also by item 1, we know that if
$\cneg \in C$, it has to exist a set $\Delta^i$ that satisfies either $\neg l$
or $l \land \pos{a} m$. This last statement is analogous for an $a$-positive
clause. Finaly, if, at any modal level \ml, we have a set $\Delta^i$ that
contains $\pos{a} m_0$ and $\nec{a} m_1, \ldots, \nec{a} m_r$, with $r \geq 0$,
we will also have a set at modal level $i+1$ that contains $m_0, m_1, \ldots,
m_r$, by item 2. 

We now specify a systematic construction procedure that, when followed,
generates either a closed or a saturated tableau for any set of clauses
$C$.

\begin{enumerate}
    \item Initially, apply the rule (PROP) to all literal clauses at all modal
        levels. This will be know as the propositional phase. If all sets of
        literals at some modal level are eliminated then return the closed
        tableau.
    \item Next, apply the rules (NEG) and (POS) to all modal clauses at all modal
        levels. This will be know as the modal phase. Again, if some modal level
        is empty, return the closed tableau.
    \item Finally, from the maximum modal level downto the first one do as
        follows: For each diamond not yet satisfied, search in the next modal
        level for a set to apply the rule (EXP). If the search fails then
        eliminate the set holding this diamond. This phase is called the
        expansion phase. If all sets at some modal level are eliminated, return
        the closed tableau. Otherwise, return the saturated tableau generated.
\end{enumerate}

There is some flexibility entailed on the procedure described in the sense that
it was left open which untreated clause to choose at the first and second
phases, and also which unsatisfied diamonds to deal with at the last phase. But
it can be shown that, no matter how the choice is made, the process must
terminate.

\begin{lemma}
    The systematic construction procedure of a saturated tableau described above
    terminates for any set of clauses $C$.
\end{lemma}
\begin{proof}
\end{proof}

\begin{theorem}
    Let $C$ be a set of clauses in \snf{K}. If $C$ is \ckn-satisfiable then it
    is \system{K}{n}{}-satisfiable.
\end{theorem}
\begin{proof}
    Let $C$ be a \ckn-satisfiable set of clauses. Following the the procedure
    described above, we are able to generate a saturated tableau for $C$. 
\end{proof}
