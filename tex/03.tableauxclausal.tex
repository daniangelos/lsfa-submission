\section{A Clausal Tableaux for \protect{\system{K}{n}{}}}
\label{sec:tableauxclausal}

Formally, a proof is a finite object constructed according to fixed set of syntactic rules that refer only to the structure of formulae, not to their intended
meaning. The set of syntactic rules that are used to provide are said to specify a \emph{calculus}. A calculus is \emph{sound} for a particular logic if any
formula that has a proof is a valid formula of this logic, and is
\emph{complete} for a logic if every valid formula has a proof~\cite{fitting}.
From these definitions, complete and sound calculi allow us to produce proofs that formulae are valid in a specific logic. As a formula is valid if, and only if, its negation is unsatisfiable, calculi which are constructed for satisfiability checking can also be used to determine whether a formula is valid or not.

Tableaux-based methods are frequently used in modal
logics because their structure is more obviously related to the notion of
possible worlds \cite{fitting}. In general, proofs are graphically represented by trees where each branch can be thought as a set of formulae. Here we present a tableaux-based calculus for deciding the satisfiability of formulae in \system{K}{n}{}. Unlike many tableaux-based systems, our system is clausal and clauses are labelled by the modal level in which a formula occur. The normal form we use is called \emph{Separated Normal Form with Modal Levels} (\snf{K}) which separates the contexts considering the different modal ineceses and different modal levels appropriately. The transformation rules and the correction proof of the translation method can be find in~\cite{nalon2015modal}.

A formula in this clausal form is represented by a set of clauses, which are
true in their respectives modal levels. A formula in \snf{K} is of the form:
$\bigwedge_i ml : C_i$,
where each $C_i$ is a clause and $ml$ is the modal level in which the clause
occurs. A clause in \snf{K} is in one of the following syntactic forms:
\begin{itemize}
    \item literal clause:
        $
        ml: \bigvee^r_{b=1} l_b
        $
    \item negative $a$-clause:
        $
        ml: l \then \pos{a}m
        $
    \item positive $a$-clause:
        $
        ml: l \then \nec{a}m
        $
\end{itemize}
where $l_b, l, m \in \Literals, a \in \Agents$ and $r,b,ml \in \Nat$. As conjunctions are associative, commutative, and idempotent, we often refer to a formula into \snf{K} as a set of clauses.

