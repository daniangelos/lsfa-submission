\section{Correctness Results}
\label{sec:correctnessresults}

\subsection{Soundness}
\label{sec:sound}

A calculus is sound when it does not prove anything that it shouldn't. A tableau
calculus for a logic $L$ containing only sound rules, with respect to $L$, is
sound~\cite{gore2009clausal}. A sound rule has a satisfiable conclusion every
time its premisses are satisfiable.

\begin{lemma}{(PROP)} Let $\gamma^i \in \cprop{k}{t}$ be a literal clause in
    \snf{ml} such that $\gamma^i \in \Gamma^i$, and $\Delta^i_j \in \Pi^i$ a
    literal set at the \ml-th modal level. If $\gamma^i$ and $\Delta^i_j$ are
    both satisfiable in \system{K}{n}{} then exists $1 \leq r \leq t$ such that
    $\Delta^i_j \cup \{l_r\}$ is also satisfiable in \system{K}{n}{}.
\end{lemma}
\begin{proof}
   Let $\Pi^i$ be the initial set of literal clauses of the \ml-th modal level,
   with $\gamma^i = \cprop{k}{t} \in \Gamma^i$, and consider $\Delta^i_j \in
   \Pi^i$ a literal set that, by hypothesis, also belongs to the \ml-th modal
   level. Supose that these hypothesis are satisfiable in \system{K}{n}{}. By
   the definition of satisfiability, there is a model $\Model = (W,
   \st_0, R_1, \ldots, R_n, \pi)$ and a world $\st \in W$ such that:
   \begin{itemize}
       \item $mdepth(\st) = \ml$ 
       \item \sat{\Model}{\st}{\bigvee_{k=1}^t l_k} and
       \item \sat{\Model}{\st}{d}, $\forall~d \in \Delta^i_j$
   \end{itemize}
   To prove that this rule is sound, it is sufficient to show that, in an
   instancy of it, at least one of its denominators is satisfiable. In fact, we
   have that:
   \begin{itemize}
       \item \sat{\Model}{\st}{\bigvee_{k=1}^t l_k} $\iff$
           \sat{\Model}{\st}{l_1} ou $\ldots$ ou \sat{\Model}{\st}{l_t}, by
           Definition~\ref{relsat}
   \end{itemize}
   This means that exists at least one index $r$, with $1 \leq r \leq t$, such
   that \sat{\Model}{\st}{l_r}. Therefore, $\Delta^i_j \cup \{l_r\}$ is
   satisfiable in \system{K}{n}{}.
\end{proof}

\begin{lemma}{(NEG)} Let $\lambda^i = \cneg$ be an $a$-negative clause in \snf{ml}
    such that $\lambda^i \in \Lambda^i$, and $\Delta^i_j \in \Pi^i$ a literal
    set in the \ml-th modal level. If $\lambda^i$ and $\Delta^i_j$ are both
    satisfiable in \system{K}{n}{} then at least one of $\Delta^i_j \cup
    \{\neg l\}$ or $\Delta^i_j \cup \{l, \pos{a}m\}$ is also satisfiable in
    \system{K}{n}{}.
\end{lemma}
\begin{proof}
   Let $\Lambda^i$ be the initial set of negative modal clauses at the \ml-th
   modal level, with $\lambda^i = \cneg \in \Lambda^i$, and consider $\Delta^i_j
   \in \Pi^i$ a literal set that, by hypothesis, also belongs to the \ml-th
   modal level. Supose that these hypothesis are all satisfiable in
   \system{K}{n}{}. By the definition of satisfiability, there exists a
   model $\Model = (W, \st_0, R_1, \ldots, R_n, \pi)$ and a world $\st \in W$
   such that:
   \begin{itemize}
       \item $mdepth(w) = \ml$ 
       \item \sat{\Model}{\st}{l \then\pos{a} m}
       \item \sat{\Model}{\st}{d}, $\forall~d \in \Delta^i_j$
   \end{itemize}
   To prove that this rule is sound, it is sufficient to show that, in an
   instancy of it, one of the denominators is satisfiable. In fact, we have
   that:
   \begin{itemize}
       \item \sat{\Model}{\st}{l \then\pos{a} m} $\iff \sat{\Model}{\st}{\neg
           l}$ ou \sat{\Model}{\st}{\pos{a}m}, by Definition~\ref{relsat}
   \end{itemize}
   If $\sat{\Model}{\st}{\neg l}$ holds then $\Delta^i_j \cup \{\neg l\}$ is
   satisfiable. Otherwise, $\langle \Model, \st \rangle \not \models \neg l$ and
   \sat{\Model}{\st}{\pos{a}m} hold, therefore, $\Delta^i_j \cup \{l, \pos{a}
   m\}$ is satisfiable in \system{K}{n}{}, by Definition~\ref{relsat}. 
\end{proof}

\begin{lemma}{(POS)} Let $\theta^i = \cpos$ be an $a$-positive clause in \snf{ml}
    such that $\theta^i \in \Theta^i$, and $\Delta^i_j \in \Pi^i$ a literal
    set in the \ml-th modal level. If $\theta^i$ and $\Delta^i_j$ are both
    satisfiable in \system{K}{n}{} then at least one of $\Delta^i_j \cup
    \{\neg l\}$ or $\Delta^i_j \cup \{l, \pos{a}m\}$ is also satisfiable in
    \system{K}{n}{}.
\end{lemma}

Being the rule (POS) quite analog to the rule (NEG), we leave the proof of its
soundness omitted.

\begin{lemma}{(ELIM1)}
    
\end{lemma}
\begin{proof}
    
\end{proof}

\begin{lemma}{(ELIM2)}
    
\end{lemma}
\begin{proof}
    
\end{proof}

\subsection{Completeness}
\label{sec:complete}
%It is complete if for all finite sets F of formulae,
%if F is C L -satisfiable then F is L -satisfiable.

A calculus is complete when it proves everything that it should. We show that
\ckn~is complete, by showing that for all sets of clauses $C$ that have an open
tableau, that is, for all sets that is \ckn-satisfiable, $C$ must also be
\system{K}{n}{}-satisfiable.

\begin{definition}
\label{def:saturated} 
    We say that a \ckn~tableau proof is \emph{saturated} if all apropriate rule
    applications have been made. More precisely, an open tableau is saturated
    if:
    \begin{enumerate}
        \item All literal clauses, $a$-positive and $a$-negative clauses have
            been treated. That is, the (PROP), (NEG) and (POS) rules have been
            aplied to all clauses in $C$, appropriately.
        \item As a result of applying the rule (ELIM1), there is no set
            containgin both $l$ and $\neg l$, for any $l$, at any modal level.
        \item At all modal levels $i$, if there is a set $\Delta^i$ that contains a
            diamond, it must be a set at modal level $i+1$ that satisfies the
            literal in the scope of the diamond and all the literals in the
            scope of the boxes that are in $\Delta^i$. Otherwise, this set would
            have been eliminated by the application of (ELIM2).
    \end{enumerate}
\end{definition}

If $\cal{T}$ is a saturated \ckn~open tableau, we can extract several useful
information. If, for instance, $\cprop{r}{k}
\in C$, we know that exists a set of literals $\Delta^i$ at modal level \ml~such
that $l_k \in \Delta^i$ for some $1 \leq k \leq r$, by item one. Also by item one,
we know that if $\cneg \in C$, it has to exist a set $\Delta^i$ that satisfies
either $\neg l$ or $l \land \pos{a} m$. This last statement is analogous for an
$a$-positive clause. Finaly, if, at any modal level \ml, we have a set
$\Delta^i$ that contains $\pos{a} m_0$ and $\nec{a} m_1, \ldots, \nec{a} m_r$,
with $r \geq 0$, we will also have a set at modal level $i+1$ that satisfies
$m_0, m_1, \ldots, m_r$, by item three. 

We now specify a systematic construction procedure that, when followed,
generates either a closed or a saturated tableau for any set of clauses
$C$.

\begin{enumerate}
    \item Initially, apply the rule (PROP) to all literal clauses at all modal
        levels, eliminating the sets where inconsistency is inserted. If all the
        sets of the initial modal level are eliminated, return a closed tableau.
    \item Next, apply the rules (NEG) and (POS) to all modal clauses at all modal
        levels, always eliminating the inconsistent sets. Again, if the first
        modal level is empty, return the closed tableau.
    \item Finally, from the maximum modal level downto the first one do as
        follows: For each diamond not yet satisfied, search in the next modal
        level for a set to satisfy the literals in the scope of this diamond and
        in the scope of every box belonging to the same set. The search fails
        when the rule (ELIM2) can be applied to the set holding this diamond.
        If all sets at the first modal level are eliminated, return the closed
        tableau. Otherwise, return the saturated tableau generated.
\end{enumerate}

There is some flexibility entailed on the procedure described in the sense that
it was left open which untreated clause to choose at the first and second
phases, and also which unsatisfied diamonds to deal with at the last phase. But
it can be shown that, no matter how the choice is made, the process must
terminate.

\begin{lemma}
    The systematic construction procedure of a saturated tableau described above
    terminates for any set of clauses $C$.
\end{lemma}
\begin{proof}
\end{proof}

\begin{theorem}
    Let $C$ be a set of clauses in \snf{K}. If $C$ is \ckn-satisfiable then it
    is \system{K}{n}{}-satisfiable.
\end{theorem}
\begin{proof}
    %Let $C$ be a \ckn-satisfiable set of clauses. Following the the procedure
    %described above, we are able to generate a saturated tableau for $C$. 
\end{proof}
