\section{Clausal Tableaux}
\label{sec:tableauxclausal}

Formally, a proof is a finite object constructed according to fixed syntactic
rules that refer only to the structure of formulae, not to their intended
meaning. The syntactic rules that define proofs are said to specify a
\emph{calculus}. A calculus is \emph{sound} for a particular logic if any
formula that has a proof must be a valid formula of this logic, and is
\emph{complete} for a logic if every valid formula has a proof~\cite{fitting}.
Then we are usually interested in sound and complete calculi as they allow us
to produce proofs that formulae are or not valid in a specific logic.

Tableaux based proof procedures are frequently (and successfully) used in modal
logics because their structure is more obviously related to that notion of
possible worlds, mentioned before~\cite{fitting}. There are many varities of
tableaux, the kind we present uses formulae in clausal normal form labelled by
their modal level. Another feature of tableux calculus is that, in general,
proofs are graphically represented by trees where each branch can be thought as
a set of formulae.

The definitions presented in this section concerning a tableau calculus
(Definition~\ref{tableauxrule} to Definition~\ref{rulecorrectness}) are
adaptations from~\cite{gore2009clausal}.

A tableau calculus for a logic $L$, denoted by \calculus{L}, is a finite set of
tableau rules as defined in Definition~\ref{tableauxrule}.

\begin{definition}
\label{tableauxrule}
    A \emph{tableau rule \trule} consists of a numerator $N$ above the line and
    a finite list of denominators $D_1, \ldots, D_k$, below the line, separated
    by vertical bars. The numerator and each denominator are a finite set of
    formulae.
\end{definition}

The numerator contains one or more distinguished formulae called \emph{principal
formulae}. All formulae in the numerator must be considered while the vertical
bars in the denominator have a disjunction meaning. This way, each rule is read
downwards as: if the set of formulae forming the numerator is satisfiable in a
logic $L$, so is at least one of the denominators.

Given a \calculus{L}, a \emph{tableau proof}, or just \emph{tableau}, for a
formula \formula~is a tree whose root carries \formula, while the other nodes
carry finite sets of formulae generated from a parent by the
application of one of the rules in \calculus{L}. A \emph{branch} in a tableau
represents a path between the tree's root and one of its nodes.

\begin{definition}
    Let $\mathcal{R}$ be a set of tableau rules. We say that $\psi$ is obtainable
    from \formula~by applications of rules from $\mathcal{R}$ if there exists a
    tableau for \formula~which uses only rules from $\mathcal{R}$ and has a
    branch that carries $\psi$.
\end{definition}

After an application of a tableau rule, it is necessary to verify if the
negation of one of the formulae recently obtained, already occurs in the
same branch. In this case, an inconsistency was inserted and a new node
containing $\bot$ is added to this branch.

\begin{definition}
    A branch in a tableau proof is \emph{closed} if its end node
    contains only $\bot$. A tableau is \emph{closed} if every one of its
    branches is closed. A tableau is \emph{open} if it is not closed.
\end{definition}

\begin{definition}
    A finite set $\mathcal{F}$ of formulae is \calculus{L}-\emph{satisfiable} if
    every \calculus{L}-tableau for $\mathcal{F}$ is open. If there is a closed
    \calculus{L}-tableau for $\mathcal{F}$ then it is
    \calculus{L}-\emph{unsatisfiable}. 
\end{definition}

\begin{definition}
    A tableau calculus \calculus{L} is \emph{sound} if for all finite sets
    $\mathcal{F}$ of formulae, if $\mathcal{F}$ is $L$-satisfiable then
    $\mathcal{F}$ is \calculus{L}-satisfiable. It is \emph{complete} if for all
    finite sets $\mathcal{F}$ of formulae, if $\mathcal{F}$ is
    \calculus{L}-satisfiable then $\mathcal{F}$ is $L$-satisfiable.
\end{definition}

\begin{definition}
\label{rulecorrectness}
    Let \trule~be a rule of \calculus{L}. We say that \trule~is sound with respect
    to $L$ if for every instance $\trule'$ of \trule, if the numerator of $\trule'$
    is $L$-satisfiable then so is one of the denominators of this instance.
\end{definition}

Any tableau calculus for a logic $L$ containing only rules sound, with respect
to $L$, is sound~\cite{gore2009clausal}.

%% MUST BE IN INTRODUCTION:
%Here, we propose a \emph{labelled tableaux calculus} for \system{K}{n}{}, which
%is called \emph{clausal tableaux calculus} since it requires clausal form.
The calculus proposed by this paper is a tableaux based proof procedure, and it
is called a \emph{labelled clausal tableaux calculus} since its rules are formed
by formulae in normal clausal form, labelled by their modal level.

There is a specific normal form to \system{K}{n}{} called \emph{Separated Normal
Form with Modal Levels} (\snf{K}) which separates the contexts considering the
different agents and different modal levels appropriately. The transformation
rules and the correction proof of the translation method can be find
in~\cite{nalon2015modal}.

A formula in this clausal form is represented by a set of clauses, which are
true in their respectives modal levels. A formula in \snf{K} is of the form:
$\bigwedge_i ml : C_i$,
where each $C_i$ is a clause and $ml$ is the modal level in which the clause
occurs.

\begin{definition}
    A \emph{literal} is a propositional symbol $p \in \Prop$ or its negation
    $\neg p$. We denote by \Literals~the set of all literals. 
\end{definition}

\begin{definition}
    A \emph{modal literal} is a formula of the form $\nec{a} l$ or its negation
    $\nec{a} \neg l$, with $l \in \Literals$ and $a \in \Agents$.
\end{definition}

Therefore, a clause in \snf{K} is in one of the following forms:
\begin{itemize}
    \item literal clause:
        $
        ml: \bigvee^r_{b=1} l_b
        $
    \item $a$-negative clause:
        $
        ml: l \then \pos{a}m
        $
    \item $a$-positive clause:
        $
        ml: l \then \nec{a}m
        $
\end{itemize}
where $l_b, l, m \in \Literals, a \in \Agents$ and $r,b,ml \in \Nat$.

%%In an implmentation level... ?
